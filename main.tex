\documentclass{article}
\usepackage{graphicx} 
\usepackage[utf8]{inputenc}
\usepackage[T1]{fontenc}
\usepackage{hyperref}
\usepackage{url}
\hypersetup{breaklinks=true}

\title{Idealno računalo}
\author{Filip Tkalčec}
\date{January 2024}

\begin{document}

\maketitle

\section{Kućište računala}
\textbf{Kućište Fractal Pop Air RGB Black TG}
\\
\\ 
\textbf{Specifikacije:}
\\Veličina matične : do ATX
\\Bočna stranica : Kaljeno staklo
\\Ugrađeni ventilatori : DA


\begin{figure}[h]
    \centering
    \includegraphics{slike/Kuciste.jpg}
    \caption{Kućište}
    \label{fig:enter-label}
\end{figure}
Navedeno kućište računala je izabrano iz razloga što bi bilo korišteno u uredu. računalo bi stajalo na stolu stoga bi njegov izgled doprinjeo izgledu radnog okruženja i time podigao motiviranost radnika na radnom mjestu.
\\ \textbf{Link :} (\url{https://www.instar-informatika.hr/kuciste-fractal-pop-air-rgb-black-tg-mid-tower-kaljeno-staklo-bez-n/120650/product/} \
\\ \textbf{Cijena : 130€}
\newpage


\section{Napajanje}
\textbf{GIGABYTE UD850GM PG5}

\\
\newline
\textbf{Specifikacije:}
\\Maksimalna izlazna snaga:850 W
\\Struje:
\\+3.3V 20 A (52.5 W)
\\+5V 20 A (52.5 W)
\\+12V 70.8 A (850 W)
\\-12V 0.3 A (3.6 W)
\\+5VSB 3 A (15 W)

Konektori:
\\20+4-pin Main Connector 1
\\4/8-pin +12V Power Connector 2
\\4-pin Peripheral Power Connector 3
\\4-pin Floppy Drive Connector 1
\\S-ATA Connector 8
\\6/8-pin PCI Express Connector 4
\\16-pin PCI Express 5.0 Connector 1

\begin{figure}[!h]
    \includegraphics[bottom]{slike/Napajanje.jpg}
    \caption{Napajanje}
    \label{fig:enter-label}
\end{figure}
Napajanje je dovoljno jako da podrži napajanje komponenti i podosta je kvalitetno.
\\ \textbf{Link :} (\url{https://www.links.hr/hr/napajanje-850w-gigabyte-ud850gm-pg5-1200-vent-80-gold-modularno-010507045}
\textbf{Cijena : 145€}

\newpage

\section{Matična ploča}
\textbf{MSI Pro B650M-A}

\\ \textbf{Specifikacije:}
\\ Shape factor µATX
\\Socket AMD AM5
\\ Chipset AMD B650
\\4x DDR5 max. 128GB

\begin{figure}[h!]
    \centering
    \includegraphics[width=100mm]{slike/Plocaa.jpg}
    \caption{Matična ploča}
    \label{fig:method}
\end{figure}

\\Matična ploča je korištena iz razloga njenih specifikacija koje će biti dovoljna da podrži ostale komponente i ona je sama koherentna sa ostalim komponentama te čini savršenu cjelinu. 
\\ \textbf{Link :}(\url{https://www.adm.hr/msi-pro-b650m-a-wifi-am5-7d77-001r/75843/product/}
\\ \textbf{Cijena: 220€}

\newpage
\section{Procesor}
\textbf{AMD Ryzen 5 7600X}
\\ \textbf{Specifikacije:}
\\ Serija - AMD Ryzen 7000 serija
\\ CPU obitelj - AMD Ryzen 5
\\ Socket - Socket-AM5
\\ Radni tak - 4.7GHz
\\ L2 predmemorija: 6 MB
\\ L3 predmemorija: 32 MB


\begin{figure}[h!]
    \centering
    \includegraphics[width=100mm]{slike/Procesor.jpg}
    \caption{Procesor}
    \label{fig:method}
\end{figure}

\\ Procesor AMD Ryzen 5 7600X je korišten u računalu jer je kompatibilan s ostalim komponentama ali isto tako je jako važan u svrsi koju će obavljatati a njegova svrha će biti procesiranje teških programa koje se koriste u poslu. Programi poput Autocad, Scada i drugi koji služe u tehničkim znanostima.

\\ \textbf{Link :}(\url{https://www.links.hr/hr/procesor-amd-ryzen-5-7600x-box-s-am5-4-7ghz-38mb-cache-6-core-bez-hladnjaka-010501004}
\\ \textbf{Cijena: 280€}


\newpage

\section{Hladnjak}
\textbf{THERMALRIGHT Phantom Spirit 120}
\\
\textbf{Specifikacije:}
\\ Veličina ventilatora - 120mm
\\Broj ventilatora - 2kom
\\Visina hladnjaka - 157mm
\\ Socket - AM5 
\\
\\
\\
\\
\\

\begin{figure}[h!]
    \centering
    \includegraphics[width=75mm]{slike/Hladnjak.jpg}
    \caption{Hladnjak}
    \label{fig:method}
\end{figure}
S obzirom da će procesoru trebati i hlađenje i dobar protok zraka u kućištu za hlađenje, biti će dovoljno i ovo hlađenje s obzirom za njegovu svrhu i brzinu rada.
\\ \textbf{Link :}(\url{https://iponcomp.hr/shop/proizvod/thermalright-phantom-spirit-120/2213537}
\\ \textbf{Cijena: 63€}

\newpage

\section{Radna memorija}
\textbf{G.Skill Ripjaws S5, 2x16GB}
\\ \textbf{Specifikacije:}
\\Vrsta memorije - DDR5
\\Kapacitet	- 32GB (2x16GB)
\\Brzina - 5600MHz
\\Latencija - CL40
\begin{figure}[h!]
    \centering
    \includegraphics[width=100mm]{slike/Memorija KDSS.jpg}
    \caption{Memorija}
    \label{fig:method}
\end{figure}
\\
Navedna memorija je korištena iz razloga što će u cjelini biti dovoljno dobra da obradi svoj dio posla i informacija. Brzina i kapacitet joj to omogućuju.

\\ \textbf{Link :}(\url{https://www.instar-informatika.hr/memorija-gskill-ripjaws-s5-32gb-2x16gb-ddr5-5600mhz-cl40/155772/product/}
\\ \textbf{Cijena: 150€}

\newpage
\section{Pohrana}
\textbf{SSD TEAM GROUP MP44L}
\\ \textbf{Specifikacije:}
\\ Prostor za pohranu :2 TB
\\ PCIe 4.0 x4 NVMe

\begin{figure}[h!]
    \centering
    \includegraphics[width=100mm]{slike/pohrana.jpg}
    \caption{Pohrana}
    \label{fig:method}
\end{figure}
\\ S obzirom na količinu podataka koje će biti smješteno na računalo potrebna je i pohrana s dosta prostora, a to nam omogućuje ovaj SSD M2. 
\\ \textbf{Link :}(\url{https://www.ronis.hr/ssd-team-group-mp44l-2-tb-pcie-40-x4-nvme/672113/product/?utm_source=nabava.net&utm_campaign=nabava.net&utm_medium=click}
\\ \textbf{Cijena: 143€}

\newpage
\section{Grafička kartica}
\textbf{Radeon RX 6950 XT}
\\ \textbf{Specifikacije:}
\\ Vrsta memorije - DDR6
\\ Radni takt (MHz)	- 2368
\\ Radna memorija (GB)- 16
\\ Memorijska sabirnica (bit) - 256


\begin{figure}[h!]
    \centering
    \includegraphics[width=150mm]{slike/Graficka.jpg}
    \caption{Grafička kartica}
    \label{fig:method}
\end{figure}

Grafička kartica je itekako potrebna u ovom projektu. Odrađivat će sve tehničke poslove.
\\ \textbf{Link :}(\url{https://www.links.hr/hr/graficka-kartica-xfx-radeon-rx-6950-xt-black-speedster-merc-319-16gb-gddr6-010503114}
\\ \textbf{Cijena: 800€}

\newpage
\section{Monitor}
\textbf{LG UltraWide 38WN95C-W}
\\ \textbf{Specifikacije:}
\\ IPS panel
\\ Veličina zaslona : 38 INCH
\\ Izvorna rezolucija : 3840x1600 (QHD+)
\\ Dimenzija slike : 21:9
\\Vrijeme odziva - 1 ms
\\Frekvencija osvježenja - 144 Hz
\\Ulazna snaga - 43 W

\begin{figure}[h!]
    \centering
    \includegraphics[width=150mm]{slike/monitor.jpg}
    \caption{Monitor}
    \label{fig:method}
\end{figure}
\\ Ovaj monitor je izrazito potreban izrazito zbog njegove veličine, kvalitete slike i dimenzije slike. 

\\ \textbf{Link :}(\url{https://www.mall.hr/monitori-27-ili-vise/lg-zakrivljeni-ultra-siroki-nano-ips-monitor-96-5-cm-qhd-38wn95c-w-?utm_source=nabava.net&utm_campaign=nabava.net&utm_medium=cse}
\\ \textbf{Cijena: 1550€}

\newpage
\section{Tipkovnica}
\textbf{Apple bežična tipkovnica Magic Keyboard}
\\ \textbf{Specifikacije:}
\\ Vrsta - s numeričkim dijelom
\\ Način povezivanja - bežični
\\Povezivost - Bluetooth, USB
\\Širina	418.7 mm
\\Visina	114.9 mm
\\Težina	390 g

\begin{figure}[h!]
    \centering
    \includegraphics[width=80mm]{slike/tipkovnica.jpg}
    \caption{Tipkovnica}
    \label{fig:method}
\end{figure}



S obzirom na posao koji će se obavljati i koliko dugo po danu te po tjednu i tako dalje, potrebna je kvalitetna i udobna tipkovnica za ured. 
\\ \textbf{Link :}(\url{https://www.mall.hr/uredske-tipkovnice/apple-bezicna-tipkovnica-magic-keyboard-hr?tab=parameters}
\\ \textbf{Cijena: 160€}


\newpage
\section{Miš}
\textbf{Miš Logitech MX Master 2S Graphite}
\\ \textbf{Specifikacije:}
\\Sučelje: Bluetooth
\\Senzor: Darkfield Laser sensor
\\Rezolucija: 4000dpi
\\Tipke: 6 + scroll
\\Baterija: Li-Po 500mAh
\begin{figure}[h!]
    \centering
    \includegraphics[width=80mm]{slike/jpg.jpg}
    \caption{Miš}
    \label{fig:method}
\end{figure}


\\ Miš je ovdje u svrhu udobnosti i značaja da ne dođe do bolova u rukama.

\\ \textbf{Link :}(\url{https://www.mikronis.hr/Proizvod/mis-logitech-mx-master-2s-graphite-p-n-910-007224-/41048}
\\ \textbf{Cijena: 109€}



 
\newpage
\section{Zvučnici}
\textbf{WHITE SHARK GSP-634}
\\ \textbf{Specifikacije:}
\\Snaga (RMS): 5 W (2 x 2.5 W)
\\Veličina zvučnika: 2 x 2 inča (4 ohm 2 x 3 W)
\\Kontrola glasnoće : Kotačić na zvučniku
\\Frekvencijski odziv: 180 Hz - 20 kHz
\\S/N omjer: >= 70 dB
\\Duljina kabela: 1.35 m
\\Napajanje: USB 5 V
\\Audio konektor: 3.5 mm stereo
\begin{figure}[h!]
    \centering
    \includegraphics[width=80mm]{slike/slika pa tako to krene.png}
    \caption{Zvučnici}
    \label{fig:method}
\end{figure}


\\ Uzeti su najobičniji zvučnici zbog jasnog zaključka, odnosno, zvučnici neće imati preveliku svrhu osim zbog glazbe. 

\\ \textbf{Link :}(\url{https://www.links.hr/hr/zvucnici-white-shark-gsp-634-flow-rgb-crni-100503713}
\\ \textbf{Cijena: 16€}

\newpage
\section{Stolica}
\textbf{Uredska stolica Ergovision Essent}
\\ \textbf{Specifikacije:}
\\Nosivost 120kg
\\Podešavanje visine naslona : 160 cm do 200

\begin{figure}[h!]
    \centering
    \includegraphics[width=80mm]{slike/erikson.jpg}
    \caption{Uredska stolica}
    \label{fig:method}
\end{figure}


\\ Zbog mnogobrojnih sati koji će biti provedeni u uredu potrebna je i udobna stolica koja neće dovesti do zamora već koja je tu da poduprije radnika za brojne sate.

\\ \textbf{Link :}(\url{https://www.instar-informatika.hr/uredska-stolica-ergovision-essent-crno-narancasta/161244/product/}
\\ \textbf{Cijena: 299€}












\newpage
\section{Cijene}
\\Ovdje su prikazane cijene svih komponenata po aktualnim tržišnim cijenama
\\ Kućište : 130€
\\ Napajanje : 145€
\\ Matična ploča : 220€
\\ Procesor : 280€
\\ Hladnjak : 83€
\\ Radna memorija : 150€
\\ Pohrana : 143€
\\ Grafička kartica : 800€
\\ Ukupna vrijednost računala : 1951€
\\ Namjena računala : Sklopovlje računala je namijenjeno za rad u raznim tehničkim područjima poput arhitekture, građevinarstva, strojarstva itd. U ovim znanostima su potrebni mnogobrojni programi poput AutoCad-a, Revit-a, Civil 3D-a itd.

\\ Nadalje su prikazane cijene ostale periferije i uređaja.
\\Monitor : 1550€
\\Tipkovnica : 160€
\\ Miš : 109€
\\Zvučnici : 16€
\\Stolica : 299€
\\Ukupna vrijednost periferije : 2134€
\\Ukupna vrijednost ureda : 4085€



\end{document}
